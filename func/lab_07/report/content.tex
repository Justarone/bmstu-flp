\chapter{Задания}

\section{Написать функцию, которая по своему аргументу-списку \texttt{lst} определяет, является ли он полиндромом (то есть равны ли \texttt{lst} и \texttt{(reverse lst)})}

\begin{lstlisting}
(defun polyndromp (lst)
  (equal lst (reverse lst)))
\end{lstlisting}

\section{Написать предикат \texttt{set-equal}, который возвращает \texttt{t}, если два его множества-аргумента содержат одни и те же элементы, порядок которых не имеет значения}

\begin{lstlisting}
(defun set-equal (lst1 lst2) 
  (and (subsetp lst2 lst1) (subsetp lst1 lst2)))
\end{lstlisting}

\section{Напишите необходимые функции, которые обрабатывают таблицу из точечных пар: \texttt{(страна . столица)}, и возвращают по стране столицу, а по столице --- страну}

\begin{lstlisting}
(defun get-cptl (cntry cntry-cptl)
  (let ((pair (assoc cntry cntry-cptl)))
    (and pair (cdr pair))))

(defun get-cntry (cptl cntry-cptl)
  (let ((pair (rassoc cptl cntry-cptl)))
    (and pair (car pair))))
\end{lstlisting}

\section{Напишите функцию \texttt{swap-first-last}, которая переставляет в списке аргументе первый и последний элементы}

\subsection{Разрушающая структуру}

\begin{lstlisting}
(defun nswap-first-last (lst)
  (let ((el1 (car lst))
        (last-el (last lst)))
    (setf (car lst) (car last-el))
    (setf (car last-el) el1)
    lst))
\end{lstlisting}

\subsection{Не разрушающая структуру}

\begin{lstlisting}
(defun swap-first-last (lst)
  (let ((el1 (car lst))
        (last-el (car (last lst))))
    (reverse (cons el1 
          (cdr 
            (reverse 
              (cons last-el (cdr lst))))))))
\end{lstlisting}

\section{Напишите функцию \texttt{swap-two-ellement}, которая переставляет в списке-аргументе два указанных своими порядковыми номерами элемента в этом списке}

\section{Разрушающая структуру}

\begin{lstlisting}
(defun nswap-two-ellement (n1 n2 lst)
  (let ((len (length lst)))
    (and (< n1 len) (< n2 len)
         (let ((el1 (nth n1 lst))
               (el2 (nth n2 lst)))
           (setf (nth n1 lst) el2)
           (setf (nth n2 lst) el1)
           lst))))
\end{lstlisting}

\section{Не разрушающая структуру}

\begin{lstlisting}
(defun swap-two-ellement (n1 n2 lst)
  (let ((len (length lst))
        (lst-copy (copy-list lst)))
    (and (< n1 len) (< n2 len)
         (let ((el1 (nth n1 lst))
               (el2 (nth n2 lst)))
           (setf (nth n1 lst-copy) el2)
           (setf (nth n2 lst-copy) el1)
           lst-copy))))
\end{lstlisting}

\section{Напишите две функции, \texttt{swap-to-left} и \texttt{swap-to-right}, которые производят круговую перестановку в списке-аргументе влево и вправо, соответственно}

\begin{lstlisting}
(defun swap-to-left (lst)
  (and lst
       (let ((tail (cdr lst))
             (head (car lst)))
         (reverse (cons head (reverse tail))))))

(defun swap-to-right (lst)
  (and lst
       (let ((last-el (car (last lst))))
         (reverse (cdr (reverse (cons last-el lst)))))))
\end{lstlisting}

\chapter{Ответы на вопросы к лабораторной работе}

\section{Способы определения функций}

%FIXME сделать

\section{Варианты и методы модификации списков}

%FIXME сделать
