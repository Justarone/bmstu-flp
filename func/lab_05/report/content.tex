\chapter{Задания}

\section{Написать функцию, которая принимает целое число и возвращает первое четное число, не меньшее аргумента}

\begin{lstlisting}
(defun first-even-ge (arg)
  (if (evenp arg) arg (+ arg 1)))
\end{lstlisting}

\section{Написать функцию, которая принимает число и возвращает число того же знака, но с модулем на 1 больше модуля аргумента}

\begin{lstlisting}
(defun module-plus (arg)
  (+ arg (if (> arg 0) 1 -1)))
\end{lstlisting}

\section{Написать функцию, которая принимает 2 число и возвращает список из этих чисел, расположенный по возрастанию}

\begin{lstlisting}
(defun growing-lst (a b)
  (if (< a b) (list a b) (list b a)))
\end{lstlisting}

\section{Написать функцию, которая принимает 3 числа и возвращает T только тогда, когда первое число расположенно между вторым и третьим}

\begin{lstlisting}
(defun pred (a b c)
  (and (> a b) (< a c)))
\end{lstlisting}

\section{Каков результат вычисления следующих выражений}

\begin{lstlisting}
(and 'fee 'fie 'foe)
\end{lstlisting}
Результат: \texttt{foe}

\begin{lstlisting}
(or 'fee 'fie 'foe)
\end{lstlisting}
Результат: \texttt{fee}

\begin{lstlisting}
(and (equal 'abc 'abc) 'foe)
\end{lstlisting}
Результат: \texttt{foe}

\begin{lstlisting}
(or Nil 'fie 'foe)
\end{lstlisting}
Результат: \texttt{fie}

\begin{lstlisting}
(and Nil 'fie 'foe)
\end{lstlisting}
Результат: \texttt{Nil}

\begin{lstlisting}
(or (equal 'abc 'abc) 'foe)
\end{lstlisting}
Результат: \texttt{T}

\section{Написать предикат, который принимает два числа-аргумента и возвращает T, если первое число не меньше второго}

\begin{lstlisting}
(defun predicate-2 (a b)
  (>= a b))
\end{lstlisting}

\section{Какой из следующих двух вариантов ошибочен и почему?}

1 вариант:
\begin{lstlisting}
(defun pred1 (x)
  (and (numberp x) (plusp x)))
\end{lstlisting}

2 вариант:
\begin{lstlisting}
(defun pred2 (x)
  (and (plusp x) (numberp x)))
\end{lstlisting}

\textbf{Ошибочен второй вариант}, потому что функция \texttt{plusp} принимает на вход один аргумент типа \texttt{number} и проверять, является ли аргумент числом, после выполнения функции \texttt{plusp} не имеет смысла, причем аргументы, не являющиеся числами, будут вызывать ошибку, в то время как 1 вариант будет работать с любым аргументом и возвращать \texttt{T} для положительных чисел.

\section{Решить задачу 4, используя для ее решения конструкции \texttt{if}, \texttt{cond}, \texttt{and/or}}

Используя \texttt{if}:
\begin{lstlisting}
(defun pred (a b c)
  (if (> a b) (< a c) Nil))
\end{lstlisting}

Используя \texttt{cond}:
\begin{lstlisting}
(defun pred (a b c)
  (cond 
  ((> a b) 
      (cond ((< a c) T) (T Nil))
  (T Nil)))
\end{lstlisting}

Используя \texttt{and/or}:
\begin{lstlisting}
(defun pred (a b c)
  (and (> a b) (< a c)))
\end{lstlisting}

\section{Переписать функцию how-alike приведенную в лекции и использующую \texttt{cond}, используя конструкции \texttt{if}, \texttt{and/or}}

Используя \texttt{cond}:
\begin{lstlisting}
(defun how-alike (x y)
  (cond ((or(= x y) (equal x y)) 'the_same)
        ((and (oddp x) (oddp y)) 'both_odd) 
        ((and (evenp x) (evenp y)) 'both_even) 
        (t 'diff)))
\end{lstlisting}

Используя \texttt{if}:
\begin{lstlisting}
(defun how-alike-if (x y)
  (if (or (= x y) (equal x y)) 'the_same
    (if (and (oddp x) (oddp y)) 'both_odd
      (if (and (evenp x) (evenp y)) 'both_even
        'diff))))
\end{lstlisting}

Используя \texttt{and/or}:
\begin{lstlisting}
(defun how-alike-and-or (x y)
  (or (and (or (= x y) (equal x y)) 'the_same)
        (and (and (oddp x) (oddp y)) 'both_odd)
        (and (and (evenp x) (evenp y)) 'both_even) 
        'diff))
\end{lstlisting}


\chapter{Ответы на вопросы к лабораторной работе}

\section{Классификация функций}

Функции в Lisp классифицируют следующим образом:

\begin{itemize}
    \item чистые математические функции;
    \item рекурсивные функции;
    \item специальные функции --- формы (сегодня 2 аргумента, завтра - 5);
    \item псевдофункции (создают эффект на внешнем устройстве);
    \item функции с вариативными значениями, из которых выбирается 1;
    \item функции высших порядков --- функционал: используется для синтаксического управления программ (абстракция языка).
\end{itemize}

По назначению функции разделяются следующим образом:

\begin{enumerate}
    \item конструкторы --- создают значение (\texttt{cons}, например);
    \item селекторы --- получают доступ по адресу (\texttt{car}, \texttt{cdr});
    \item предикаты --- возвращают \texttt{Nil}, \texttt{T}.
\end{enumerate}

\section{Работа функций \texttt{and}, \texttt{or}, \texttt{if}, \texttt{cond}}

FIXME

\section{Способы определения функций}

FIXME (см. лр 4)
