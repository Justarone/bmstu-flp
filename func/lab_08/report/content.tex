\chapter{Задания}

\section{Напишите функцию, которая умножает на заданное число-аргумент все числа из заданного списка-аргумента, когда...}

\subsection{а) все элементы списка --- числа}

\begin{lstlisting}
(defun mult-all-numbers (mult lst)
  (mapcar #'(lambda (el) (* el mult)) lst))
\end{lstlisting}

\subsection{б) элементы списка --- любые объекты}

\begin{lstlisting}
(defun compl-mult-all-numbers (mult lst)
  (mapcar #'(lambda (el) 
              (cond ((listp el) (compl-mult-all-numbers mult el))
                    (T (* el mult))))
          lst))
\end{lstlisting}

\section{Напишите функцию, \texttt{select-between}, которая из списка-аргумента, содержащего только числа, выбирает только те, которые расположены между двумя указанными границами-аргументами и возвращает их в виде списка (упорядоченного по возрастанию списка чисел)}

\begin{lstlisting}
(defun get-n (n lst acc)
  (cond ((or (null lst) (<= n 0)) (reverse acc))
  (T (get-n (- n 1) (cdr lst) (cons (car lst) acc)))))

(defun select-between (from to lst) 
    (sort (get-n (+ (- to from) 1) (nthcdr from lst) Nil) #'<))
\end{lstlisting}

\section{Что будет результатом \texttt{(mapcar 'вектор '(570-40-8))}?}

Данная программа завершится с ошибкой по причине того, что функции \texttt{вектор} не существует.

\section{Напишите функцию, которая уменьшает на 10 все числа из списка аргумента этой функции}

\begin{lstlisting}
(defun lst-minus-10 (lst)
  (mapcar #'(lambda (x) (- x 10)) lst))
\end{lstlisting}

\section{Написать функцию, которая возвращает первый аргумент списка-аргумента, который сам является непустым списком}

\begin{lstlisting}
(defun first-sublist (lst)
  (and lst (cond ((consp (car lst)) (car lst))
                 (T (first-sublist (cdr lst))))))
\end{lstlisting}

\section{Найти сумму числовых элементов смешанного структурированного списка}

\begin{lstlisting}
(defun count-all-in-list (lst)
  (reduce #'(lambda (acc el)
              (cond ((listp el) (+ acc (count-all-in-list el)))
                    ((numberp el) (+ acc el))
                    (T acc)))
          (cons 0 lst)))
\end{lstlisting}

\chapter{Ответы на вопросы к лабораторной работе}

\section{Порядок работы и варианты использования функционалов}

Функционалы --- функции, которые в качестве одного из аргументов используют другую функцию (специальным образом).

\subsection{Применяющие функционалы} 

Данные функционалы просто позволяют применить переданную в качестве аргумента функцию к переданным в качестве аргументов параметрам.

Виды:
\begin{enumerate}
    \item \texttt{funcall} (вызывает функцию-аргумент с остальными аргументами);
        
        Синтаксис: \texttt{(funcall \#'fun arg1 arg2 ... argN)}

        Пример: \texttt{(funcall \#'+ 1 2 3)}

    \item \texttt{apply} (вызывает функцию-аргумент с аргументами из списка, переданного вторым аргументом в \texttt{apply}).

        Синтаксис: \texttt{(apply \#'fun arg-lst)}

        Пример: \texttt{(apply \#'+ '(1 2 3))}
\end{enumerate}

\subsection{Отображающие функционалы} 

Отображения множества аргументов в множество значений, позволяют многократно применить функцию, некоторый аналог цикла из императивных языков.

Данные функции берут аргумент, являющийся функцональным объектом (функцией), и многократно применяет эту фукнцию к элементам переданного в качестве аргумента списка.

\begin{enumerate}
    \item \texttt{mapcar};

        Сначала функция \texttt{fun} применяется ко всем первым элементам списков-аргументов, затем ко всем вторым аргументам и так до тех пор, пока не кончатся элементы самого короткого списка. К полученным результатам применения функции применяется функция \texttt{list}, поэтому на выходе функции всегда будет список.

        Синтаксис: \texttt{(mapcar \#'fun lst1 lst2 ... lstN)}

        Пример: \texttt{(mapcar \#'(lambda (x y) (+ x y)) '(1 2 3) '(6 5 4)) -> (list (+ 1 6) (+ 2 5) (+ 2 4))}

    \item \texttt{maplist};

        Работает похожим на \texttt{mapcar} образом, но в качестве аргумента на каждой итерации функция \texttt{fun} получает хвост списка, который использовался на предыдущей итерации (изначально функция получает сам список-аргумент). Если функция принимает несколько аргументов и передано несколько аргументов-списков, то они передаются функции \texttt{fun} в том же порядке, в которым идут в \texttt{maplist}.

        Синтаксис: \texttt{(maplist \#'fun lst1 lst2 ... lstN)}

        Пример: \texttt{(maplist \#'(lambda (x y) (+ (car x) (car y))) '(1 2 3 4) '(6 5 4)) -> (list (+ 1 6) (+ 2 5) (+ 2 4))}

    \item \texttt{mapcan};

        Работает аналогично \texttt{mapcar}, только соединяет результаты функции с помощью функции \texttt{nconc}. Может использоваться как \texttt{filter-map} из некоторых современных языков (например, функция, которая оставляет только четные числа и возводит их в квадрат)

        Синтаксис: \texttt{(mapcan \#'fun lst1 lst2 ... lstN)}

        Пример: \texttt{(mapcan \#'(lambda (x) (and (oddp x) (list (* x x)))) '(1 2 3 4 5 6 7 8 9)) -> (1 9 25 49 81)}

    \item \texttt{mapcon};

        Работает аналогично \texttt{maplist}, только соединяет результаты функции с помощью функции \texttt{nconc}.

        Синтаксис: \texttt{(mapcon \#'fun lst1 lst2 ... lstN)}

    \item \texttt{find-if};

        Возвращает первый элемент списка, для которого функция-предикат возвращает не \texttt{Nil}.

        Синтаксис: \texttt{(find-if \#'predicat lst)}

        Пример: \texttt{(find-if \#'oddp '(2 4 1)) -> 1}

    \item \texttt{remove-if}, \texttt{remove-if-not};

        Данные функции возвращают список, в котором находятся только те элементы, для которых функция-предикат вернула не \texttt{Nil} (для \texttt{remove-if-not} вернула \texttt{Nil}).

        Синтаксис: \texttt{(remove-if \#'predicat lst)}

        Пример: \texttt{(remove-if \#'oddp '(1 2 3 4 5 6)) -> (2 4 6)}

    \item \texttt{reduce};

        Применяет функцию к элементам списка каскадно. ''Накапливает значение'', применяя функцию-аргумент к результату предыдущей итерации и следующему элементу списка (изначально инициализирует результат первым элементом, в случае пустого списка пытается вызвать функцию-аргумент без аргументов и вернуть значение)

        Синтаксис: \texttt{(reduce \#'aggregator lst)}

        Пример: \texttt{(reduce \#'oddp '(1 2 3 4 5 6)) -> (2 4 6)}

    \item \texttt{every};

        Возвращает \texttt{T}, если функция-предикат возвращает не \texttt{Nil}, для всех элементов списка-аргумента.

        Синтаксис: \texttt{(every \#'predicat lst)}

        Пример: \texttt{(every \#'oddp '(1 3 5 7)) -> T}

    \item \texttt{some};

        Возвращает \texttt{T}, если функция-предикат возвращает не \texttt{Nil}, хотя бы для одного элемента списка-аргумента.

        Синтаксис: \texttt{(some \#'predicat lst)}

        Пример: \texttt{(some \#'oddp '(1 2 3 4 5)) -> T}
\end{enumerate}
