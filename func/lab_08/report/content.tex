\chapter{Задания}

\section{Напишите функцию, которая умножает на заданное число-аргумент все числа из заданного списка-аргумента, когда...}

\subsection{а) все элементы списка --- числа}

\begin{lstlisting}
(defun mult-all-numbers (mult lst)
  (mapcar #'(lambda (el) (* el mult)) lst))
\end{lstlisting}

\subsection{б) элементы списка --- любые объекты}

\begin{lstlisting}
(defun compl-mult-all-numbers (mult lst)
  (mapcar #'(lambda (el) 
              (cond ((listp el) (compl-mult-all-numbers mult el))
                    (T (* el mult))))
          lst))
\end{lstlisting}

\section{Напишите функцию, \texttt{select-between}, которая из списка-аргумента, содержащего только числа, выбирает только те, которые расположены между двумя указанными границами-аргументами и возвращает их в виде списка (упорядоченного по возрастанию списка чисел)}

\begin{lstlisting}
(defun get-n (n lst acc)
  (cond ((or (null lst) (<= n 0)) (reverse acc))
  (T (get-n (- n 1) (cdr lst) (cons (car lst) acc)))))

(defun select-between (from to lst) 
    (sort (get-n (+ (- to from) 1) (nthcdr from lst) Nil) #'<))
\end{lstlisting}

\section{Что будет результатом \texttt{(mapcar 'вектор '(570-40-8))}?}

Данная программа завершится с ошибкой по причине того, что функции \texttt{вектор} не существует.

\section{Напишите функцию, которая уменьшает на 10 все числа из списка аргумента этой функции}

\begin{lstlisting}
(defun lst-minus-10 (lst)
  (mapcar #'(lambda (x) (- x 10)) lst))
\end{lstlisting}

\section{Написать функцию, которая возвращает первый аргумент списка-аргумента, который сам является непустым спском}

\begin{lstlisting}
(defun first-sublist (lst)
  (and lst (cond ((listp (car lst)) (car lst))
                 (T (first-sublist (cdr lst))))))
\end{lstlisting}

\section{Найти сумму числовых элементов смешанного структурированного списка}

\begin{lstlisting}
(defun count-all-in-list (lst)
  (reduce #'(lambda (acc el)
              (cond ((listp el) (+ acc (count-all-in-list el)))
                    ((numberp el) (+ acc el))
                    (T acc)))
          (cons 0 lst)))
\end{lstlisting}

\chapter{Ответы на вопросы к лабораторной работе}

\section{Порядок работы и варианты использования функционалов}

%FIXME сделать
