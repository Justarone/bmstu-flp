\chapter{Задания}

\section{Написать рекурсивную версию (с именем \texttt{rec-add}) вычисления суммы чисел заданного списка}

\begin{lstlisting}
(defun rec-add-internal (lst acc)
  (cond ((null lst) acc)
        (T (rec-add-internal (cdr lst) (+ acc (car lst))))))
(defun rec-add (lst)
  (rec-add-internal lst 0))
\end{lstlisting}

\section{Написать рекурсивную функцию с именем \texttt{rec-nth} функции \texttt{nth}}

\begin{lstlisting}
(defun rec-nth (n lst)
  (and lst (cond ((zerop n) (car lst))
                 (T (rec-nth (- n 1) (cdr lst))))))
\end{lstlisting}

\section{Написать рекурсивную функцию \texttt{alloddr}, которая возвращает \texttt{t}, когда все элементы списка нечётные}

\begin{lstlisting}
(defun alloddr (lst)
  (or (null lst) 
      (and (oddp (car lst))
           (alloddr (cdr lst)))))
\end{lstlisting}

\section{Написать рекурсивную функцию, относящуюся к хвостовой рекурсии с одним тестом завершения, которая возвращает последний элемент списка-аргумента}

%FIXME Что за "один тест завершения"?... Надо уточнить!

\begin{lstlisting}
(defun last-rec (lst)
  (cond ((null (cdr lst)) (car lst))
        (T (last-rec (cdr lst)))))
\end{lstlisting}

\section{Написать рекурсивную функцию, относящуюся к дополняемой рекурсии с одним тестом завершения, которая вычисляет сумму всех чисел от 0 до \textit{n}-ого аргумента функции}

%FIXME кто блин такая ваша "дополняемая рекурсия"?

\subsection{От \textit{n}-аргумента функции, до последнего $>= 0$}

\subsection{От \textit{n}-аргумента функции до \textit{m}-аргумента с шагом \textit{d}}

\section{Написать рекурсию, которая возвращает последнее нечетное число из числового списка, возможно создавая некоторые вспомогательные функции}

\begin{lstlisting}
(defun last-odd-internal (lst cur-odd)
  (cond ((null lst) cur-odd)
        (T (cond ((oddp (car lst)) (last-odd-internal (cdr lst) (car lst)))
                 (T (last-odd-internal (cdr lst) cur-odd))))))
(defun last-odd (lst)
  (last-odd-internal lst nil))
\end{lstlisting}

\section{Используя \texttt{cons}-дополняемую рекурсию с одним тестом завершения, написать функцию, которая получает как аргумент список чисел, а возвращает список квадратов этих чисел в том же порядке}

%FIXME ну мне определенно не помешает где-то разузнать, что это за рекурсии такие, но очевидно, что здесь такое решение должно быть..
\begin{lstlisting}
(defun cons-square (lst)
  (and lst (cons ((lambda (x) (* x x)) (car lst))
                 (con-square (cdr lst)))))
\end{lstlisting}

\section{Написать функцию с именем \texttt{select-odd}, которая из заданного списка выбирает все нечетные числа}

Варианты:
\begin{enumerate}
    \item select-even;
    \item вычисляет сумму всех нечетных чисел (sum-all-odd) или сумму всех четных чисел из заданного списка.
\end{enumerate}

\begin{lstlisting}
; reverses the list
(defun my-filter-reducer (filter reducer lst init-el)
    (reduce #'(lambda (acc x)
                (cond ((funcall filter x) (funcall reducer acc x))
                      (T acc)))
            (cons init-el lst)))

(defun my-filter (func lst)
  (reverse (my-filter-reducer func #'(lambda (acc x) (cons x acc)) lst nil)))

(defun select-odd (lst)
  (my-filter #'oddp lst))
(defun select-even (lst)
  (my-filter #'evenp lst))

(defun sum-all-odd (lst)
  (my-filter-reducer #'oddp #'+ lst 0))
(defun sum-all-even (lst)
  (my-filter-reducer #'evenp #'+ lst 0))
\end{lstlisting}


\section{Создать и обработать смешанный структурированный список с информацией:}

%FIXME сделать

\begin{itemize}
    \item ФИО;
    \item зарплата;
    \item возраст;
    \item категория (квалификация).
\end{itemize}

Изменить зарплату в зависимости от заданного условия, и подсчитать суммарную зарплату. Использовать композиции функций.
