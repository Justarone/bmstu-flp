\chapter{Задания}

\section{Чем принципиально отличаются функции \texttt{cons}, \texttt{list}, \texttt{append}? Пусть...}

\textbf{Продолжение задания:} 

Пусть
\begin{lstlisting}
(setf lst1 '(a b))
(setf lst2 '(c d))
\end{lstlisting}

Каковы результаты следующих выражений?

\noindent\rule{12cm}{0.4pt}

\begin{lstlisting}
(cons lst1 lst2)
\end{lstlisting}
Результат: \texttt{((A B) C D)}

\begin{lstlisting}
(list lst1 lst2)
\end{lstlisting}
Результат: \texttt{((A B) (C D))}

\begin{lstlisting}
(append lst1 lst2)
\end{lstlisting}
Результат: \texttt{(A B C D)}

\section{Каковы результаты следующих выражений?}

\begin{lstlisting}
(reverse ())
\end{lstlisting}
Результат: \texttt{Nil}

\begin{lstlisting}
(last ())
\end{lstlisting}
Результат: \texttt{Nil}

\begin{lstlisting}
(reverse '(a))
\end{lstlisting}
Результат: \texttt{(a)}

\begin{lstlisting}
(last '(a))
\end{lstlisting}
Результат: \texttt{(a)}

\begin{lstlisting}
(reverse '((a b c)))
\end{lstlisting}
Результат: \texttt{((a b c))}

\begin{lstlisting}
(last '((a b c)))
\end{lstlisting}
Результат: \texttt{((a b c))}


\section{Написать, по крайней мере, два варианта функции, которая возвращает последний элемент своего списка-аргумента}

\subsection{Рекурсия 1}

\begin{lstlisting}
(defun my-last-recursive-internal (lst)
  (if (cdr lst) 
   (my-last-recursive-internal (cdr lst))
   (car lst)))
(defun my-last-recursive (lst)
  (and lst (my-last-recursive-internal lst)))
\end{lstlisting}

\subsection{Рекурсия 2}

\begin{lstlisting}
(defun my-last-recursive-internal-2 (lst last-el)
  (if (eql nil lst) last-el (my-last-recursive-internal-2 (cdr lst) (car lst))))
(defun my-last-recursive-2 (lst)
  (my-last-recursive-internal-2 lst nil))
\end{lstlisting}

\subsection{С помощью \texttt{reduce}}

\begin{lstlisting}
(defun my-last-reduce (lst)
  (reduce #'(lambda (acc el) el) lst))
\end{lstlisting}

\section{Написать, по крайней мере, два варианта функции, которая возвращает свой список-аргумент без последнего элемента}

\subsection{Рекурсия}

\begin{lstlisting}
(defun no-last-internal (lst acc)
  (if (cdr lst)
   (no-last-internal (cdr lst) (cons (car lst) acc))
   (nreverse acc)))
(defun no-last (lst)
  (and lst (no-last-internal lst nil)))
\end{lstlisting}

\subsection{Функции ядра}

\begin{lstlisting}
(defun no-last-kern (lst)
  (and lst (nreverse (cdr (reverse lst)))))
\end{lstlisting}


\section{Написать простой вариант игры в кости, в котором...}

\textbf{Продолжение задания:}

в котором бросаются две правильные кости. Если сумма выпавших очков равна 7 или 11 --- выигрыш, если выпало $(1, 1)$ или $(6, 6)$ --- игрок получает право снова бросить кости, во всех остальных случаях ход переходит ко второму игроку, но запоминается сумма выпавших очков. Если второй игрок не выигрывает абсолютно, то выигрывает тот игрок, у которого больше очков. Результат игры и значения выпавших костей выводить на экран с помощью функции \texttt{print}.

\noindent\rule{12cm}{0.4pt}

\begin{lstinputlisting}[
        caption={Игра в кости},
        style={mystyle},
    ]{../dices.lisp}
\end{lstinputlisting}

\chapter{Ответы на вопросы к лабораторной работе}

\section{Структуроразрушающие и не разрушающие структуру списка функции}

\subsection{Не разрушающие структуру списка функции}

Данные функции не меняют сам объект-аргумент, а создают копию.

\subsubsection{Функция \texttt{append}}

Объединяет списки. Это форма, можно передать больше 2 аргументов. Создает копию для всех аргументов, кроме последнего.

Пример: \texttt{(append '(1 2) '(3 4))} --- \texttt{(1 2 3 4)}.

\subsubsection{Функция \texttt{reverse}}

Возвращает копию исходного списка, элементы которого переставлены в обратном порядке. \textbf{В целях эффективности работает только на верхнем уровне}.

Пример: \texttt{(reverse '(1 2 3 4))} --- \texttt{(4 3 2 1)}.

\subsubsection{Функция \texttt{last}}

Проход по верхнему уровню и возврат последней списковой ячейки.

Пример: \texttt{(last '(1 2 3 4))} --- \texttt{(4)}.

\subsubsection{Функция \texttt{nth}}

Возврат указателя от n-ной списковой ячейки, нумерация с нуля.

Пример: \texttt{(nth 1 '(1 2 3 4))} --- \texttt{2}.

\subsubsection{Функция \texttt{nthcdr}}

Возврат n-ого хвоста.

Пример: \texttt{(nthcdr 1 '(1 2 3 4))} --- \texttt{(2 3 4)}.

\subsubsection{Функция \texttt{length}}

Возврат длины списка (\textbf{только по верхнему уровню}).

Пример: \texttt{(length '(1 2 (3 4)))} --- \texttt{3}.

\subsubsection{Функция \texttt{remove}}

Модифицирует, но работает с копией, поэтому не разрушает. Данная функция удаляет элемент по значению (Часто разрушающая аналогичная функция называется \texttt{delete}). По умолчанию используется \texttt{eql} для сравнения на равенство, но можно передать другую функцию через ключевой параметр \texttt{:test}.

Примеры: 
\begin{enumerate}
    \item \texttt{(remove 3 '(1 2 3))} --- \texttt{(1 2)};
    \item \texttt{(remove '(1 2) '((1 2) (3 4)))} --- \texttt{((1 2) (3 4))};
    \item \texttt{(remove '(1 2) '((1 2) (3 4)) :test #'equal)} --- \texttt{((3 4))};
\end{enumerate}

\subsubsection{Функция \texttt{rplaca}}

Переставляет \texttt{car}-указатель на 2 элемент-аргумент (\textit{S}-выражение).

Пример: \texttt{(rplaca '(1 2 3) 3)} --- \texttt{(3 2 3)}.

\subsubsection{Функция \texttt{rplacd}}

Переставляет \texttt{cdr}-указатель на 2 элемент-аргумент (\textit{S}-выражение).

Пример: \texttt{(rplacd '(1 2 3) '(4 5))} --- \texttt{(1 4 5)}.

\subsubsection{Функция \texttt{subst}}

Заменяет все элементы списка, которые равны 2 переданному элементу-аргументу на другой 1 элемент-аргумент. \textit{По умолчанию для сравнения используется функция \texttt{eql}}.

Пример: \texttt{(subst 2 1 '(1 2 1 3))} --- \texttt{(2 2 2 3)}.

\subsection{Структуроразрушающие функции}

Данные функции меняют сам объект-аргумент, невозможно вернуться к исходному списку. Чаще всего такие функции начинаются с префикса \texttt{n-}.

\subsubsection{Функция \texttt{nconc}}

Работает аналогично \texttt{append}, только не копирует свои аргументы, а разрушает структуру.

\subsubsection{Функция \texttt{nreverse}}

Работает аналогично \texttt{reverse}, но не создает копии.

\subsubsection{Функция \texttt{nsubst}}

Работае аналогично функции \texttt{nsubst}, но не создает копии.

\section{Отличие в работе функций \texttt{cons}, \texttt{list}, \texttt{append} и в их результате}

Функция \texttt{cons} --- чисто математическая, конструирует списковую ячейку, которая может вовсе и не быть списком (будет списком только в том случае, если 2 аргументом передан список).

Примеры:
\begin{enumerate}
    \item \texttt{(cons 2 '(1 2))} --- \texttt{(2 1 2)} --- список;
    \item \texttt{(cons 2 3)} --- \texttt{(2 . 3)} --- не список.
\end{enumerate}

Функция \texttt{list} --- форма, принимает произвольное количество аргументов и конструирует из них список. Результат --- всегда список. При нуле аргументов возвращает пустой список.

Примеры:
\begin{enumerate}
    \item \texttt{(list 1 2 3)} --- \texttt{(1 2 3)};
    \item \texttt{(list 2 '(1 2))} --- \texttt{(2 (1 2))};
    \item \texttt{(list '(1 2) '(3 4))} --- \texttt{((1 2) (3 4))};
\end{enumerate}

Функция \texttt{append} --- форма, принимает на вход произвольное количество аргументов и для всех аргументов, кроме последнего, создает копию, ссылая при этом последний элемент каждого списка-аргумента на первый элемент следующего по порядку списка-аргумента (так как модифицируются все списки-аргументы, кроме последнего, копирование для последнего не делается в целях эффективности).

Примеры:
\begin{enumerate}
    \item \texttt{(append '(1 2) '(3 4))} --- \texttt{(1 2 3 4)};
    \item \texttt{(append '((1 2) (3 4)) '(5 6))} --- \texttt{((1 2) (3 4) 5 6)}.
\end{enumerate}
