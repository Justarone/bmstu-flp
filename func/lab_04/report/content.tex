\chapter{Задания}

\section{Написать функцию, которая переводи температуру в системе Фаренгейта в температуру в системе по Цельсию \texttt{(defun f-to-c (temp) ... )}}

\begin{lstlisting}
(defun f-to-c (temp) 
  (* (/ 5 9) (- temp 32.0)))
\end{lstlisting}

\section{Что получится при вычислении каждого из выражений}

\begin{lstlisting}
(list 'cons T Nil)
\end{lstlisting}
Результат: \texttt{(cons T Nil)}

\begin{lstlisting}
(eval (eval (list 'cons T Nil)))
\end{lstlisting}
Результат: ошибка: после применения внутреннего \texttt{eval} вычисленное значение будет \texttt{(T)}. Затем попытка применить \texttt{eval} к полученному заканчивается ошибкой.

\begin{lstlisting}
(apply #'cons '(T Nil))
\end{lstlisting}
Результат: \texttt{(T)}

\begin{lstlisting}
(eval (list 'cons T Nil))
\end{lstlisting}
Результат: \texttt{(T)}

\begin{lstlisting}
(list 'eval Nil)
\end{lstlisting}
Результат: \texttt{(eval Nil)}

\begin{lstlisting}
(eval Nil)
\end{lstlisting}
Результат: \texttt{Nil}

\begin{lstlisting}
(eval (list 'eval Nil))
\end{lstlisting}
Результат: \texttt{Nil}

\clearpage

\section{Написать функцию, вычисляющую катет по заданной гипотенузе и другому катету прямоугольного треугольника, и составить диаграмму ее вычисления.}

\begin{lstlisting}
(defun leg (h l)
  (sqrt (- (* h h) (* l l))))
\end{lstlisting}

\vspace{60mm}

\section{Написать функцию, вычисляющую площадь трапеции по ее основаниям и высоте, и составить диаграмму ее вычисления.}

\begin{lstlisting}
(defun area (a b h) 
  (* (/ (+ a b) 2) h))
\end{lstlisting}

\chapter{Ответы на вопросы к лабораторной работе}

\section{Синтаксическая форма и хранение программы в памяти}

FIXME

\section{Трактовка элементов списка}

FIXME (здесь надо написать, что первый аргумент --- имя функции, остальные --- аргументы этой функции)

\section{Порядок реализации программы}

FIXME

\section{Способы определения функций}

FIXME

