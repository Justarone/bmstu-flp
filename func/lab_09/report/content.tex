\chapter{Задания}

\section{Написать функцию, которая выбирает из заданного списка только ты числа, которые больше 1 и меньше 10 (Вариант: между 2 заданными границами)}

\textbf{Реализация для случая, когда между 2 заданными границами}

\subsection{С помощью функционалов}

\begin{lstlisting}
(defun list-between-only (from to lst)
  (mapcan #'(lambda (el)
              (and (< el to) (> el from) (list el)))
          lst))
\end{lstlisting}

\subsection{Рекурсия}

\begin{lstlisting}
(defun list-btwn-only-internal (from to lst acc)
  (cond ((null lst) acc)
        (T (let ((head (car lst)))
          (cond ((and (< head to) (> head from)) 
                 (list-btwn-only-internal from to (cdr lst) (cons head acc)))
                (T (list-btwn-only-internal from to (cdr lst) acc)))))))
(defun list-btwn-only-rec (from to lst)
  (reverse (list-btwn-only-internal from to lst nil)))
\end{lstlisting}

\section{Написать функцию, вычисляющую декартово произведение двух своих списков-аргументов}

\subsection{С помощью функционалов}

\begin{lstlisting}
(defun list-mul (lst1 lst2)
  (mapcan #'(lambda (outer-el)
              (mapcar #'(lambda (inner-el)
                          (cons outer-el inner-el))
                      lst2))
          lst1))
\end{lstlisting}

\subsection{Рекурсия}
\begin{lstlisting}
(defun list-mul-internal (cur-lst1 src-lst1 lst2 acc)
  (cond ((null lst2) acc)
        ((null cur-lst1) (list-mul-internal src-lst1 src-lst1 (cdr lst2) acc))
        (T (list-mul-internal (cdr cur-lst1) src-lst1 lst2 (cons (cons (car cur-lst1) (car lst2)) acc)))))
(defun list-mul-rec (lst1 lst2)
  (and lst1 lst2 (reverse (list-mul-internal lst1 lst1 lst2 nil))))
\end{lstlisting}


\section{Почему так реализовано \texttt{reduce}, в чем причина?}

\texttt{(reduce \#'+ ()) -> 0}

Поведение в данном примере обусловлено работой функции \texttt{+}. Эта функция --- функционал, который при 0 количестве аргументов возвращает значение 0. Если подать на вход \texttt{reduce} функцию, которая не может обработать 0 аргументов (например, математическая функция \texttt{cons}), то вызов \texttt{reduce} с пустым списком в качестве второго аргумента вернет ошибку (\texttt{invalid number of arguments: 0}). При этом, если подано более одного аргумента, то \texttt{reduce} выполняет следующие действия:
\begin{enumerate}
    \item сохраняет первый элемент списка в область памяти (для определенности назовем ее \texttt{acc});
    \item для всех остальных элементов списка выполняет переданную в качестве первого аргумента функцию, подавая на вход 2 аргумента (\texttt{acc} и очередной элемент списка) и сохраняя результат в \texttt{acc}.
\end{enumerate}

Пример упрощенной реализации \texttt{reduce} (в данной реализации опущены проверки аргументов):

\begin{lstlisting}
(defun my-reduce-internal (func lst acc)
  (cond ((null lst) acc)
        (T (my-reduce-internal func (cdr lst) (funcall func acc (car lst))))))
(defun my-reduce (func lst)
  (cond ((null lst) (funcall func))
        (T (my-reduce-internal func (cdr lst) (car lst)))))
\end{lstlisting}

\section{Пусть \texttt{list-of-lists} --- список, состоящий из списков. Написать функцию, которая вычисляет сумму длин всех элементов \texttt{list-of-lists}, то есть, например, для аргумента \texttt{((1 2) (3 4)) -> 4}}

\subsection{С помощью функционалов}

\begin{lstlisting}
(defun sum-lens (list-of-lists)
  (reduce #'(lambda (acc lst) (+ acc (length lst)))
          (cons 0 list-of-lists)))
\end{lstlisting}

\subsection{Рекурсия}

\begin{lstlisting}
(defun sum-lens-rec-internal (lol acc)
  (cond ((null lol) acc)
        (T (sum-lens-rec-internal (cdr lol) (+ acc (length (car lol)))))))
(defun sum-lens-rec (list-of-lists)
  (and (listp list-of-lists) (sum-lens-rec-internal list-of-lists 0)))
\end{lstlisting}

\section{Используя рекурсию, написать функцию, которая по исходному списку стоит список квадратов чисел смешанного структурированного списка}

\begin{lstlisting}
(defun square-list (lst)
  (cond ((null lst) lst)
        ((atom lst) (* lst lst))
        (T (cons (square-list (car lst))
                 (square-list (cdr lst))))))
\end{lstlisting}

\chapter{Ответы на вопросы к лабораторной работе}

\section{Классификация рекурсивных функций}

\textbf{Рекурсия} --- ссылка на описываемый объект во время его описания.

Классификация рекурсивных функций:
\begin{itemize}
    \item Простая (рекурсивный вызов --- единственный);
    \item Второго порядка (несколько рекурсивных вызовов);
    \item Взаимная рекурсия (используются несколько рекурсивных функций, которые могут друг друга вызывать).
    \item Хвостовая рекурсия (при очередном вызове рекурсивной функции все действия до входа выполнены, а при выходе ничего более делать не приходится);
    \item Дополняемая рекурсия (результат рекурсии используется, как аргумент некоторой другой функции (которую называют \textit{дополняемой функцией}); частный случай --- \texttt{cons}-дополняемая рекурсия).
\end{itemize}


