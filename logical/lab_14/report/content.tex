\chapter{Лабораторная работа №14}

\textbf{Задание:} используя базу знаний, хранящую знания (лаб. 13):
\begin{itemize}
    \item \textbf{<<Телефонный справочник>>:} Фамилия, №тел, Адрес - структура (Город, Улица, №дома, №кв),
    \item \textbf{<<Автомобили>>:} Фамилия\_владельца, Марка, Цвет, Стоимость, и др.,
    \item \textbf{<<Вкладчики банка>>:} Фамилия, Банк, счет, сумма, др.
\end{itemize}

Владелец может иметь несколько телефонов, автомобилей, вкладов (Факты). В разных городах есть однофамильцы, в одном городе --- фамилия уникальна.

Используя \textbf{Коньюктивное правило и простой вопрос}, обеспечить возможность поиска:

По Марке и Цвету автомобиля найти Фамилию, Город, Телефон и Банки, в которых владелец автомобиля имеет вклады. Лишей информации не находить и не передавать!!!

Владельцев может быть \textbf{несколько} (не более 3-х), \textbf{один} и \textbf{ни одного}.

\begin{enumerate}
    \item Для каждого из трёх вариантов \textbf{словесно подробно} описать порядок формирования ответа (в виде таблицы). При этом, указать – отметить моменты очередного запуска алгоритма унификации и полный результат его работы. Обосновать следующий шаг работы системы. Выписать унификаторы – подстановки. Указать моменты, причины и результат отката, если он есть.
    \item Для случая нескольких владельцев (2-х): приведите примеры (таблицы) работы системы \textbf{при разных порядках} следования в БЗ процедур, и знаний в них: (\textbf{<<Телефонный справочник>>, <<Автомобили>>, <<Вкладчики банков>>}, или: \textbf{<<Автомобили>>, <<Вкладчики банков>>, <<Телефонный справочник>>})) Сделайте вывод: Одинаковы ли: множество работ и объём работ в разных случаях?
    \item Оформите 2 таблицы, демонстрирующие \textbf{порядок работы алгоритма унификации} вопроса и подходящего заголовка правила (для двух случаев из пункта 2) и укажите результаты его работы: ответ и побочный эффект.
\end{enumerate}

\begin{lstlisting}
domains
  surname = string
  city, street = string
  house, flat = integer
  phone = string
  address = addr(city, street, house, flat)
  mark = string
  color = string
  price = integer
  bank = string
  id, amount = integer
  
predicates
  phone(surname, phone, address)
  car(surname, mark, color, price)
  bank_depositor(surname, bank, id, amount)
  man_by_car(mark, color, surname, city, phone, bank)
  
clauses
  phone("Perestoronin", "+79999999999", addr("Moscow", "Lesnaya", 12, 2)).
  phone("Romanov", "+71111111111", addr("Moscow", "Lesnaya", 13, 87)).
  phone("Nitenko", "+73333333333", addr("Ekaterinburg", "Kamennaya", 13, 87)).
  phone("Yacuba", "+66666666666", addr("Moscow", "Wall-street", 123, 87)).
  car("Nitenko", "bmw", "green", 1000).
  car("Romanov", "bmw", "green", 1000).
  car("Yacuba", "volkswagen", "red", 10000).
  car("Yacuba", "golfR", "black", 20000).
  car("Romanov", "bike", "white", 10).
  car("Perestoronin", "mercedes", "yellow", 30000).
  bank_depositor("Nitenko", "Sber", 22, 1000).
  bank_depositor("Yacuba", "Sber", 33, 10000).
  bank_depositor("Yacuba", "Alfa", 44, 20000).
  bank_depositor("Romanov", "Sper", 238, 10).
  bank_depositor("Perestoronin", "Maze", 1, 10000).
  
  man_by_car(Mark, Color, Surname, City, Phone, Bank) :- 
      car(Surname, Mark, Color, _),
      phone(Surname, Phone, addr(City, _, _, _)),
      bank_depositor(Surname, Bank, _, _).
  
goal
    man_by_car("bmw", "green", Surname, City, Phone, Bank).
    %man_by_car("volkswagen", "red", Surname, City, Phone, Bank).
    %man_by_car("ford", "mustang", Surname, City, Phone, Bank).
\end{lstlisting}

\textbf{Таблицы для заданий 1-3 представлены на отдельных листах и приложены к отчету.}

\chapter{Лабораторная работа №15}


\textbf{Задание:} создать базу знаний \textbf{<<Собственники>>}, дополнив базу знаний, хранящую знания (лаб. 13):

\begin{itemize}
    \item \textbf{<<Телефонный справочник>>}: Фамилия, №тел, Адрес – структура (Город, Улица, №дома, №кв),
    \item \textbf{<<Автомобили>>}: Фамилия\_владельца, Марка, Цвет, Стоимость, и др.,
    \item \textbf{<<Вкладчики банков>>}: Фамилия, Банк, счёт, сумма, др.
\end{itemize}

знаниями о дополнительной \textbf{собственности} владельца. \textbf{Преобразовать} знания об автомобиле к форме знаний о собственности.

Вид собственности (кроме автомобиля):
\begin{itemize}
    \item \textbf{Строение, стоимость} и другие его характеристики;
    \item \textbf{Участок, стоимость} и другие его характеристики;
    \item \textbf{Водный\_транспорт, стоимость} и другие его характеристики.
\end{itemize}

Описать и использовать вариантный домен: \textbf{Собственность}. Владелец может иметь, но \textbf{только} один объект \textbf{каждого вида собственности} (это касается и \textbf{автомобиля}), или не иметь некоторых видов собственности.
Используя \textbf{конъюнктивное правило и разные формы задания} \textbf{одного вопроса} (пояснять для какого №задания – какой вопрос), обеспечить возможность поиска:
\begin{enumerate}
    \item Названий всех объектов собственности заданного субъекта,
    \item Названий и стоимости всех объектов собственности заданного субъекта,
    \item Разработать правило, позволяющее найти суммарную стоимость всех объектов собственности заданного субъекта.
\end{enumerate}
Для 2-го пункта и \textbf{одной} фамилии \textbf{составить таблицу}, отражающую конкретный порядок работы системы, с объяснениями порядка работы и особенностей использования доменов (указать конкретные T1 и T2 и полную подстановку на каждом шаге)

\begin{lstlisting}
domains
  surname = string
  city, street = string
  house, flat = integer
  phone = string
  address = addr(city, street, house, flat)
  mark = string
  color = string
  price = integer
  bank = string
  id, amount = integer
  name = string
  ind_property = building(name, price);
    region(name, price);
    water_transport(mark, color, price);
    car(mark, color, price).
  
predicates
  phone(surname, phone, address)
  bank_depositor(surname, bank, id, amount)
  owner(surname, ind_property)

  all_objects(surname, name)
  all_objects_with_price(surname, name, price)
  
clauses
  phone("Perestoronin", "+79999999999", addr("Moscow", "Lesnaya", 12, 2)).
  phone("Romanov", "+71111111111", addr("Moscow", "Lesnaya", 13, 87)).
  phone("Nitenko", "+73333333333", addr("Ekaterinburg", "Kamennaya", 13, 87)).
  phone("Yacuba", "+66666666666", addr("Moscow", "Wall-street", 123, 87)).

  owner("Nitenko", car("bmw", "green", 1000)).
  owner("Nitenko", region("empty field", 1000)).
  owner("Nitenko", building("Moscow center", 1000)).
  owner("Romanov", car("bmw", "green", 1000)).
  owner("Romanov", region("rublevka", 10000)).
  owner("Romanov", building("mini-village", 20000)).
  owner("Romanov", water_transport("bmw", "red", 10000)).
  owner("Yacuba", car("golfR", "black", 20000)).
  owner("Yacuba", building("tiktok", 200000)).
  owner("Perestoronin", car("mercedes", "yellow", 30000)).
  owner("Perestoronin", building("tent", 10)).

  bank_depositor("Nitenko", "Sber", 22, 1000).
  bank_depositor("Yacuba", "Sber", 33, 10000).
  bank_depositor("Yacuba", "Alfa", 44, 20000).
  bank_depositor("Romanov", "Sper", 238, 10).
  bank_depositor("Perestoronin", "Maze", 1, 10000).

  all_objects(Surname, Name) :- owner(Surname, car(Name, _, _)).
  all_objects(Surname, Name) :- owner(Surname, building(Name, _)).
  all_objects(Surname, Name) :- owner(Surname, region(Name, _)).
  all_objects(Surname, Name) :- owner(Surname, water_transport(Name, _, _)).

  all_objects_with_price(Surname, Name, Price) :- owner(Surname, car(Name, _, Price)).
  all_objects_with_price(Surname, Name, Price) :- owner(Surname, building(Name, Price)).
  all_objects_with_price(Surname, Name, Price) :- owner(Surname, region(Name, Price)).
  all_objects_with_price(Surname, Name, Price) :- owner(Surname, water_transport(Name, _, Price)).
goal
  all_objects("Romanov", Name).
  %all_objects_with_price("Perestoronin", Name, Price).
\end{lstlisting}

\textbf{Таблица для 2-го пункта представлена на отдельных листах и приложена к отчету.}
