\chapter{Лабораторная работа №16}

\textbf{Задание:} создать базу знаний: <<ПРЕДКИ>>, позволяющую наиболее эффективным способом (за меньшее количество шагов, что обеспечивается меньшим количеством предложений БЗ – правил), и используя разные варианты (примеры) одного вопроса, определить (указать: какой вопрос для какого варианта):

\begin{enumerate}
    \item По имени субъекта определить всех его бабушек (предки 2-го колена);
    \item По имени субъекта определить всех его дедушек (предки 2-го колена);
    \item По имени субъекта определить всех его бабушек и дедушек (предки 2-го колена);
    \item По имени субъекта определить его бабушку по материнской линии (предки 2-го колена);
    \item По имени субъекта определить его бабушку и дедушку по материнской линии (предки 2-го колена).
\end{enumerate}

Минимизировать количество правил и количество вариантов вопросов. Использовать конъюнктивные правила и простой вопрос.

Для одного из вариантов ВОПРОСА и конкретной БЗ составить таблицу, отражающую конкретный порядок работы системы, с объяснениями:

\begin{itemize}
    \item очередная проблема на каждом шаге и метод ее решения,
    \item каково новое текущее состояние резольвенты, как получено,
    \item какие дальнейшие действия? (запускается ли алгоритм унификации? Каких термов? Почему этих?),
    \item вывод по результатам очередного шага и дальнейшие действия.
\end{itemize}

Так как резольвента хранится в виде стека, то состояние резольвенты требуется отображать в столбик: вершина – сверху! Новый шаг надо начинать с нового состояния резольвенты!

\begin{lstlisting}
domains
  sex = symbol
  name = string
  man = man(sex, name)
  
predicates
  parent(man, man)
  grandparent(man, sex, name)
  
clauses
  grandparent(man(Sex, Gname), Tsex, Name) :- parent(man(Sex, Gname), man(Tsex, Tname)),
                         parent(man(Tsex, Tname), man(_, Name)).

  parent(man(f, "Lena"), man(m, "Pasha")).
  parent(man(m, "Gena"), man(m, "Pasha")).
  parent(man(m, "Vitaly"), man(m, "Gena")).
  parent(man(f, "Natalia"), man(m, "Gena")).
  parent(man(m, "Anatoly"), man(f, "Lena")).
  parent(man(f, "Lyalya"), man(f, "Lena")).

goal
  grandparent(man(f, Gname), _, "Pasha").
  %grandparent(man(m, Gname), _, "Pasha").
  %grandparent(man(_, Gname), _, "Pasha").
  %grandparent(man(f, Gname), f, "Pasha").
  %grandparent(man(_, Gname), f, "Pasha").
\end{lstlisting}

\textbf{Таблицы представлены на отдельных листах и приложены к отчету.}

\chapter{Лабораторная работа №17}


\textbf{Задание:} в одной программе написать правила, позволяющие найти
\begin{enumerate}
    \item Максимум из двух чисел:
    \begin{itemize}
        \item Без использования отсечения;
        \item С использованием отсечения;
    \end{itemize}
    \item Максимум из трех чисел:
    \begin{itemize}
        \item Без использования отсечения;
        \item С использованием отсечения.
    \end{itemize}
\end{enumerate}

Убедиться в правильности результатов. Для каждого случая из пункта 2 обосновать необходимость всех условий тела. Для одного из вариантов ВОПРОСА и каждого варианта задания 2 составить таблицу, отражающую конкретный порядок работы системы.

Так как резольвента хранится в виде стека, то состояние резольвенты требуется отображать в столбик: вершина – сверху! Новый шаг надо начинать с нового состояния резольвенты!

Требуется ответить на вопрос: <<За счет чего может быть достигнута эффективность работы системы?>>

\begin{lstlisting}
domains
  num = integer

predicates
  max2(num, num, num)
  max3(num, num, num, num)
  max2short(num, num, num)
  max3short(num, num, num, num)
  
clauses
  max2(N1, N2, N1) :- N1 >= N2.
  max2(N1, N2, N2) :- N2 >= N1.

  max2short(N1, N2, N1) :- N1 >= N2, !.
  max2short(_, N2, N2).

  max3(N1, N2, N3, N1) :- N1 >= N2, N1 >= N3.
  max3(N1, N2, N3, N2) :- N2 >= N1, N2 >= N3.
  max3(N1, N2, N3, N3) :- N3 >= N1, N3 >= N2.

  max3short(N1, N2, N3, N1) :- N1 >= N2, N1 >= N3, !.
  max3short(_, N2, N3, N2) :- N2 >= N3, !.
  max3short(_, _, N3, N3).

goal
  max2(1, 2, Max).
  %max2short(2, 1, Max).
  %max3(4, 2, 3, Max).
  %max3short(4, 2, 3, Max).
\end{lstlisting}

\textbf{Таблицы представлены на отдельных листах и приложены к отчету.}
